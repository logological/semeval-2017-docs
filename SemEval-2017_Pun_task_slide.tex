\documentclass[
   utf8x,
   accentcolor=tud9c,
   landscape,
   colorbacktitle,
   inverttitle,
   presentation,
   t,
   override,
   svgnames
]{tudbeamer}

% A slide for the SemEval-2017 shared task on puns, presented
% at SemEval-2016
%
% Typeset by Tristan Miller <http://www.nothingisreal.com/>
% June 2016

\usepackage[canadian]{babel}
\usepackage[utf8x]{inputenc}

\title{SemEval-2017 Shared Task: Detection and Interpretation of English Puns}
\institute{Tristan Miller, Iryna Gurevych (UKP Lab, Technische Universität Darmstadt); Christian F.~Hempelmann (Texas A\&M University–Commerce)}
\date{17 June 2016}

\begin{document}

\begin{frame}{SemEval-2017 Shared Task:\newline Detection and Interpretation of English Puns}
  \begin{itemize}
  \item Traditional WSD assumes each word in context carries a single meaning
  \item Puns are (usually humorous) constructions that exploit lexical polysemy or phonological similarity to suggest \emph{two} contrasting meanings:
\end{itemize}

\begin{center}
\emph{When the church bought gas for their annual barbecue,\\ proceeds went from the sacred to the propane.}
\end{center}

\begin{itemize}
  \item Shared task: Automatically detect the presence of puns in short English texts, and annotate their double meanings with WordNet sense keys
  \item Downstream applications: humanization of natural language interfaces, machine-assisted translation, sentiment analysis, digital humanities
  \item Task is suitable for those with an interest or experience in lexical semantics (WSD, SRL), phonology, discourse processing, humour studies, etc.
  \end{itemize}
\end{frame}

\end{document}

%%% Local Variables:
%%% mode: latex
%%% TeX-master: t
%%% End:
